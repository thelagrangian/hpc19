\documentclass[amsmath,amssymb]{revtex4}
\usepackage{amssymb}
\usepackage{amsmath}
\usepackage[amsmath,thmmarks]{ntheorem}
\usepackage{graphicx}% Include figure files
\usepackage{dcolumn}% Align table columns on decimal point
\usepackage{bm}% bold math
\usepackage{longtable}
%\usepackage{slashbox}
\usepackage[colorlinks]{hyperref}
\setcounter{section}{-1}

\begin{document}

\title{Spring 2019: Advanced Topics in Numerical Analysis:\\
High Performance Computing\\
Assignment 3}
\author{Yongyan Rao, yr780@nyu.edu}
%\email{yr780@nyu.edu}
\date{\today}
\maketitle

\section{\label{sec:sec0}CPU specification}
The processors used in the experiment are located in one of Courant's servers, crunchy5.cims.nyu.edu, with the following specs. The server has $4$ AMD Opteron(TM) Processor 6272 processors, each processor has 8 cores, and each core supports 2 threads i.e., 2 virtual cores. Therefore, from the software point of view, the server has $4\times8\times2 = 64$ cores.
\begin{longtable}{l l}
    \hline 
processor         &: 0 to 63\\ \hline
vendor\_id	&: AuthenticAMD \\ \hline
cpu family	&: 21\\ \hline
model		&: 1\\ \hline
model name	&: AMD Opteron(TM) Processor 6272\\ \hline
stepping	&: 2\\ \hline
microcode	&: 0x600063e\\ \hline
cpu MHz		&: 2100.039\\ \hline
cache size	&: 2048 KB\\ \hline
physical id	&: 0 to 3\\ \hline
siblings	&: 16\\ \hline
core id		&: 0 to 7\\ \hline
cpu cores	&: 8\\ \hline
apicid		&: {\it omitted}\\ \hline
initial apicid	&: {\it omitted}\\ \hline
fpu		&: yes\\ \hline
fpu\_exception	&: yes\\ \hline
cpuid level	&: 13\\ \hline
wp		&: yes\\ \hline
flags		&: {\it omitted}\\ \hline
bogomips	&: 4200.07, 4199.44, 4199.47, 4199.45\\ \hline
TLB size	&: 1536 4K pages\\ \hline
clflush size	&: 64\\ \hline
cache\_alignment	&: 64\\ \hline
address sizes	&: 48 bits physical, 48 bits virtual\\ \hline
power management&: {\it omitted}\\ \hline
\caption{CPU specs.}
\end{longtable}

\section{\label{sec:sec1}1. Approximating Special Functions Using Taylor Series \& Vectorization}
Both {\tt sin4\_intrin()} and {\tt sin4\_vec()} functions have been updated to achieve 12-digit accuracy.

One can also implement fast sine function with argument whose magnitude is greater than $\frac{\pi}{4}$. Assume $x\in \mathbb{R}$, there always exist $k\in \mathbb{Z}$ and $-\frac{\pi}{4}\le\theta\le\frac{\pi}{4}$, such that $x = \theta + \frac{\pi}{2}k$. Therefore, $\sin{x} = \sin{(\theta +\frac{\pi}{2}k)}$
\begin{equation}
=
\begin{cases}
  \sin{\theta}& {\rm when~}k\equiv0\pmod 4,\\
  \cos{\theta}& {\rm when~}k\equiv1\pmod 4,\\
  -\sin{\theta}& {\rm when~}k\equiv2\pmod 4,\\
  -\cos{\theta}& {\rm when~}k\equiv3\pmod 4.\\
\end{cases}
\end{equation}
It indicates that to have a complete implementation of fast sine function, sine and cosine functions need to be implemented together. Cosine function can be approximated by the following Taylor sum to achieve the same accuracy.
\begin{equation}
\cos{x} = 1 - \frac{1}{2}x^2 + \frac{1}{4\cdot3\cdot2}x^4 - \frac{1}{6\cdot5\cdot4\cdot3\cdot2}x^6 + \frac{1}{8\cdot7\cdot6\cdot5\cdot4\cdot3\cdot2}x^8 - \frac{1}{10\cdot9\cdot8\cdot7\cdot6\cdot5\cdot4\cdot3\cdot2}x^{10}, \mid x\mid\le\frac{\pi}{4}.
\end{equation}
The complete implementation is not included in the code submission.

\section{\label{sec:sec2}2. Parallel Scan in OpenMP}
The server used in the experiment has $4$ AMD Opteron(TM) Processor 6272 processors, each processor has 8 cores, and each core supports 2 threads i.e., 2 virtual cores. Therefore, from the software point of view, the server has $4\times8\times2 = 64$ cores. Experiments were conducted to find the relation between speedup and the number of threads. Six (6) sets of experiments were conducted with the following numbers of threads, 2, 4, 8, 16, 32, and 64. For each set, the number of threads was fixed, and the experiments were repeated five (5) times. The relation between average speedup and the number of threads is represented as below, and the original experiment results are also attached.

\begin{longtable}{c | c | c | c | c | c | c }
    \hline 
Num. of threads& 2 & 4 & 8 & 16 & 32 & 64  \\ \hline \hline
Speedup		&3.420	&6.105 &	7.777 & 7.649 & 7.880 & 7.575\\\hline
\caption{Average speedup versus number of threads.}
\end{longtable}

From the result above, one can conclude that for problem size $N = 100,000,000$, the program's performance does not improve significantly with the increase of number of threads after the number is greater than 8.

\begin{longtable}{c| c | c }
    \hline 
     Num. of threads & OpenMP time (second) & Sequential time (second)   \\ \hline \hline
2	&0.323469	&1.137197	\\\hline
2	&0.344838	&1.136338	\\\hline
2	&0.325571	&1.139111	\\\hline
2	&0.344009	&1.136943	\\\hline
2	&0.325147	&1.137792	\\\hline

4	&0.190642	&1.137009	\\\hline
4	&0.180924	&1.136997	\\\hline
4	&0.190120	&1.137396	\\\hline
4	&0.179606	&1.139102	\\\hline
4	&0.190271	&1.136423	\\\hline

8	&0.134885	&1.137976	\\\hline
8	&0.171541	&1.137122	\\\hline
8	&0.132695	&1.139145	\\\hline
8	&0.173160	&1.136448	\\\hline
8	&0.119127	&1.137515	\\\hline

16	&0.164737	&1.139703	\\\hline
16	&0.124929	&1.137433	\\\hline
16	&0.167803	&1.137495	\\\hline
16	&0.123704	&1.136592	\\\hline
16	&0.162812	&1.139350	\\\hline

32	&0.131675	&1.139518	\\\hline
32	&0.163054	&1.137547	\\\hline
32	&0.132328	&1.139154	\\\hline
32	&0.164675	&1.136779	\\\hline
32	&0.130386	&1.137027	\\\hline

64	&0.157351	&1.139402	\\\hline
64	&0.160024	&1.137056	\\\hline
64	&0.137521	&1.138797	\\\hline
64	&0.140094	&1.138478	\\\hline
64	&0.156504	&1.138708	\\\hline
\caption{Time measures (second) for parallel and sequential scanning.}
\end{longtable}




\end{document}

\documentclass[amsmath,amssymb]{revtex4}
\usepackage{amssymb}
\usepackage{amsmath}
\usepackage[amsmath,thmmarks]{ntheorem}
\usepackage{graphicx}% Include figure files
\usepackage{dcolumn}% Align table columns on decimal point
\usepackage{bm}% bold math
\usepackage{longtable}
%\usepackage{slashbox}
\usepackage[colorlinks]{hyperref}
%\setcounter{section}{-1}

\begin{document}

\title{Spring 2019: Advanced Topics in Numerical Analysis:\\
High Performance Computing\\
Assignment 5}
\author{Yongyan Rao, yr780@nyu.edu}
%\email{yr780@nyu.edu}
\date{\today}
\maketitle


\section{\label{sec:sec1}1. MPI ring communication}
The integer ring for measuring latency and the array ring for measuring bandwidth are implemented in two (2) separate files, {\tt int\_ring.c} and {\tt array\_ring.c}.

Note: In the implementation of the array ring (for measuring bandwidth), the operation of adding process rank is only applied to the first element (with index 0) of the array, rather than the entire array. The operation, on one hand, helps error check, and on the other hand, avoids the additional iteration over each element of the array.
\begin{itemize}
\item Measured on crunchy3, with 20 processes, the time latency is $0.000644$ ms.
\item Measured between crunchy1 and crunchy3, with 20 processes, the time latency is $0.005395$ ms.
\item Measured on crunchy3, with 20 processes, the network bandwidth is $5.880489\times10^{-2}$ GB/s.
\item Measured between crunchy1 and crunchy3, with 20 processes, the network bandwidth is $3.325312\times10^{-2}$ GB/s.
\end{itemize}


\section{\label{sec:sec2}2. Final project}
\begin{center}
  \begin{tabular}{|c|p{10cm}|p{3cm}|}
    \hline
    \multicolumn{3}{|c|}{\bf Project: Implementing FFT} \\
    \hline
    Week & Work & Who  \\ \hline \hline
    04/15-04/21 & Literature research on FFT and its algorithms & Yongyan Rao \\ \hline
    04/22-04/28 & Further literature research, implemented a sequential version of FFT, checked the correctness by comparing with GSL library & Yongyan Rao \\ \hline
    04/29-05/05 &  & Yongyan Rao\\ \hline
    05/06-05/12 &  & Yongyan Rao \\ \hline
    05/13-05/19 &   & Yongyan Rao \\ \hline
  \end{tabular}
  \end{center}





\end{document}

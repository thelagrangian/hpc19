\documentclass[amsmath,amssymb]{revtex4}
\usepackage{amssymb}
\usepackage{amsmath}
\usepackage[amsmath,thmmarks]{ntheorem}
\usepackage{graphicx}% Include figure files
\usepackage{dcolumn}% Align table columns on decimal point
\usepackage{bm}% bold math
\usepackage{longtable}
%\usepackage{slashbox}
\usepackage[colorlinks]{hyperref}
%\setcounter{section}{-1}
\usepackage{pgfplots}
%\usepackage{caption}
%\usepackage{subcaption}
\usepackage{tikz}
\usetikzlibrary{plotmarks}

\usepackage{listings}
\usepackage{xcolor} % for setting colors





\begin{document}

\title{Spring 2019: Advanced Topics in Numerical Analysis:\\
High Performance Computing\\
Assignment 6}
\author{Yongyan Rao, yr780@nyu.edu}
%\email{yr780@nyu.edu}
\date{\today}
\maketitle


\section{\label{sec:sec0}0. Final project}
\begin{center}
  \begin{tabular}{|c|p{10cm}|p{3cm}|}
    \hline
    \multicolumn{3}{|c|}{\bf Project: Implementing FFT} \\
    \hline
    Week & Work & Who  \\ \hline \hline
    04/15-04/21 & Literature research on FFT and its algorithms & Yongyan Rao \\ \hline
    04/22-04/28 & Further literature research, implemented a sequential version of FFT, checked the correctness by comparing with GSL library & Yongyan Rao \\ \hline
    04/29-05/05 & Implemented OpenMP version and naive cuda version of FFT & Yongyan Rao\\ \hline
    05/06-05/12 & Ran tests on the implementations, and worked on report & Yongyan Rao \\ \hline
    05/13-05/19 &   & Yongyan Rao \\ \hline
  \end{tabular}
  \end{center}


\section{\label{sec:sec1}1. MPI-parallel two-dimensional Jacobi smoother}
\subsection{\label{sec:sec1.1}Blocking version weak scaling study}
The following study was conducted with the following parameters, $lN=100$, number of iterations $10$.
\begin{center}
  \begin{tabular}{c | c | c | c | c | c}
    \hline
Number of process(es) & 1  & 4 & 16 & 64 &256     \\ \hline
Time            (second)    & 0.000687 & 0.000895 & 0.099191 & 0.064470 & 3.127207 \\ \hline
  \end{tabular}
\end{center}

\begin{figure}[h]
  
  \centering
\begin{tikzpicture}
\begin{axis}[
  legend pos=north west,
  legend cell align=left,
  xlabel=Problem size,
  ylabel=Time (s),
  xmode=log,
  ymode=log,
  ]


\addplot [mark=o, color=blue, mark size=2pt]
table {./plot/weak.plot};
\addlegendentry[font=\tiny]{Running time}

\end{axis}
\end{tikzpicture}
\caption{Weak scaling}
\end{figure}


\subsection{\label{sec:sec1.2}Blocking version strong scaling study}
The following study was conducted with the following parameters, $N=\sqrt{10240000} = 3200$, number of iterations $10$.
\begin{center}
  \begin{tabular}{c | c | c | c | c | c}
    \hline
Number of process(es) & 1  & 4 & 16 & 64 &256     \\ \hline
Time        (second)  & 0.673925 & 0.171769 & 0.085054 & 0.117849 & 3.683604 \\ \hline
Speedup                        & 1 &3.923437873&7.923495662&5.718546615&0.182952619\\ \hline
  \end{tabular}
\end{center}

\begin{figure}[h]
  
  \centering
\begin{tikzpicture}
\begin{axis}[
  legend pos=north west,
  legend cell align=left,
  xlabel=Problem size,
  ylabel=Time (s),
  xmode=log,
  ymode=log,
  ]


\addplot [mark=o, color=blue, mark size=2pt]
table {./plot/strong.plot};
\addlegendentry[font=\tiny]{speedup}

\addplot [mark=o, color=red, mark size=2pt]
table {./plot/ref.plot};
\addlegendentry[font=\tiny]{Ideal speedup}

\end{axis}
\end{tikzpicture}
\caption{Strong scaling}
\end{figure}



\subsection{\label{sec:sec1.3}Non-blocking version comparison}
The same experiments were conducted using the non-blocking implementation.
\begin{center}
  \begin{tabular}{c | c | c | c | c | c}
    \hline
Number of process(es) & 1  & 4 & 16 & 64 &256     \\ \hline
Blocking time (second)    & 0.000687 & 0.000895 & 0.099191 & 0.064470 & 3.127207 \\ \hline
Non-blocking time (second) & 0.000673 & 0.000764&  0.020252& 0.042427 & 0.717789\\ \hline
  \end{tabular}
\end{center}

\begin{center}
  \begin{tabular}{c | c | c | c | c | c}
    \hline
Number of process(es) & 1  & 4 & 16 & 64 &256     \\ \hline
Blocking time        (second)  & 0.673925 & 0.171769 & 0.085054 & 0.117849 & 3.683604 \\ \hline
Non-blocking time (second) & 0.624653 & 0.160637& 0.066820 &  0.057231& 0.705940 \\ \hline
  \end{tabular}
\end{center}

\section{\label{sec:sec2}2. Parallel sample sort}
With setup, {\tt --nodes=8} and {\tt --ntasks-per-node=8}, 

\begin{center}
  \begin{tabular}{c | c | c | c }
    \hline
N & $10^4$  & $10^5$ & $10^6$    \\ \hline
Time (second) & 0.802799 & 0.851473&  1.159685\\ \hline
  \end{tabular}
\end{center}


\end{document}
